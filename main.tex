% 以下导言区内容将得到保留,直至版面设计基本固定,以后将会封装至sty内


% 指定格式命令:
% \bm 矢量标示
% \md 关键词标注
% \textbf 加粗标注
% \margin 边注
% \cf 图片引用


\documentclass[10pt,a4paper]{book}
\usepackage{babel}
\usepackage{ctex}
\usepackage[dvipsnames, svgnames, x11names]{xcolor}
\usepackage{amsmath}
\usepackage{amsfonts}
\usepackage{amssymb}%数学符号宏包
\usepackage{geometry}
\usepackage{fancyhdr}
\usepackage{framed}
\usepackage{fontspec}
\usepackage{ntheorem}
\usepackage{nomencl}
\usepackage{nicematrix}
\usepackage{multicol}
\usepackage{indentfirst}%首行缩进
\setlength{\parindent}{2em}
\usepackage[bookmarks=true,colorlinks,linkcolor=black]{hyperref}
\usepackage{CJKfntef}
\usepackage{amsmath,bm}%重要宏包,粗斜体\bm
\usepackage{makeidx}%重要宏包,用于添加索引
\usepackage{pgf,tikz,pgfplots}%一般绘图宏包
\pgfplotsset{compat=1.15}
\usepackage{mathrsfs}
\usetikzlibrary{arrows}
\usepackage{tikz-3dplot}%3d绘图宏包
\usepackage{tcolorbox}%box宏包
\usepackage{subfigure}
\usepackage{autobreak}
\usepackage{enumerate}
\usepackage{wasysym}
\usepackage{textcomp}
\usepackage{marginnote}
\usepackage{tikz}
\usetikzlibrary{shapes.geometric}
\usetikzlibrary{arrows,arrows.meta}
\usepackage{ulem}%下划线宏包用法和样式如下:
%\uuline{双下划线}
%\uwave{波浪线}
%\sout{中间删除线}
%\xout{斜删除线}
%\dashuline{虚线}
%\dotuline{加点}
\usepackage{titletoc}%目录页的宏包
\usepackage[center]{titlesec}%改变章节或标题的样式的宏包
	\usepackage{pifont} 
	\usepackage[perpage]{footmisc}  %每页脚注重新编号
	\renewcommand{\thefootnote}{\normalsize \ding{\numexpr191+\value{footnote}}}  %使用pifont包里\
\makenomenclature
%\titleformat{command}[shape]{format}{label}{sep}{before}[after]

%1.command 是要重新定义的各种标题命令,比如 \part,\chapter,\section,\subsection,\subsubsection,\paragraph,\subparagraph等;
%2.shape 是用来设定段落形状的,可选的参数有 hang 、 block 、 display 等,详见 titlesec 文档,位于:TEXLIVE/VERSION/texmf-dist/doc/latex/titlesec
%3.format 用于定义标题外观,比如使标题居中、字体加粗等;
%4.label 用于定义定义标题的标签,就是标题内容前面的标号;
%5.sep 定义标题的标签与标题内容之间的间隔距离。
%6.before 用于在标题内容前再加些内容;
%7.after 用于在标题内容后再加些内容。

\usetikzlibrary{shapes,arrows}
\makeindex%添加索引
%缺省页面
\geometry{inner=2cm,outer=2cm,bottom=1.5cm,top=2cm,marginparwidth=3.5cm,marginparsep=0.5cm,includemp}
%\geometry{showframe,showcrop}%此句用于显示排版文本框

%目录调整
\newcounter{mycontents}
\newcommand{\thecontents}{\refstepcounter{mycontents} \alph{mycontents}.}
%\titlecontents{标题名}[左间距]{标题格式}{标题标志}{无序号标题}{指引线与页码}[下间距]
\titlecontents{chapter}
[0cm]
{\bf \large \vspace{0.8em} }{\contentspush{第 \thecontentslabel\ 章 \hspace*{0.8em}}}{}{\titlerule*[0.5pc]{$\cdot$}\contentspage}
\titlecontents{section}[1.7cm]{\bf  \vspace{0.5em} }{\contentslabel{2.4em}}{\hspace*{-2.5em} \thecontents \hspace*{0.8em}}{\titlerule*[0.5pc]{$\cdot$}\contentspage}
\titlecontents{subsection}[2.5cm]{\small \vspace{0.2em} }{\contentslabel{3em}}{}{\titlerule*[0.5pc]{$\cdot$}\contentspage}

%字体设置-------------------------------
\setCJKmainfont[BoldFont={Source Han Sans HC-Medium}]{Source Han Serif CN-Regular}
\setCJKfamilyfont{song}{Source Han Serif SC-Regular}
\setCJKfamilyfont{hei}{Source Han Sans HC-Regular}
\setCJKfamilyfont{heiti}{Source Han Sans HC-Medium}
\setCJKfamilyfont{heilight}{Source Han Sans HC-Light}
\setCJKfamilyfont{title}{Source Han Sans HC-Regular}
\setCJKfamilyfont{songbold}{Source Han Serif CN-SemiBold}
\setCJKfamilyfont{kaishu}{STKaiti}
\theorembodyfont{\CJKfamily{song}}
%\setmainfont{Times New Roman}

%颜色设置-——————————
\definecolor{f8766d}{HTML}{EA7500}
\definecolor{1a9850}{HTML}{0097e6}
\definecolor{ffa725}{HTML}{1289A7}
\definecolor{2a7ae2}{HTML}{7158e2}
\definecolor{6a3d9a}{HTML}{ED4C67}
\definecolor{53a9ab}{HTML}{007500}
\definecolor{titlepurple}{HTML}{5758BB}
\definecolor{titlepurpleb}{HTML}{833471}
\definecolor{titlepurplec}{HTML}{006266}
\definecolor{md}{HTML}{EA2027}
\definecolor{background}{HTML}{f5f5ed}
\definecolor{white}{HTML}{F8F8FF}
\definecolor{文字底色}{HTML}{F8FF00}%强调的文字底色(目前:黄色)

%geogebra颜色
\definecolor{zzttqq}{rgb}{0.6,0.2,0}
\definecolor{uuuuuu}{rgb}{0.26666666666666666,0.26666666666666666,0.26666666666666666}
\definecolor{ududff}{rgb}{0.30196078431372547,0.30196078431372547,1}
\definecolor{xdxdff}{rgb}{0.49019607843137253,0.49019607843137253,1}

%定理定义环境设置--------------------

%\newtcbox{\mybox}[1][]{on line,
%	arc=0pt,outer arc=0pt,colback=#1!10!white,colframe=#1,
%	boxsep=0pt,left=3pt,right=3pt,top=2pt,bottom=2pt,
%	boxrule=0pt,leftrule=1.5pt}
\newcommand{\mybox}[2][]{
	\begin{tcolorbox}[on line,
		arc=0pt,outer arc=0pt,colback=#1!7!white,colframe=#1,
		boxsep=0pt,left=3pt,right=3pt,top=3pt,bottom=3pt,
		boxrule=0pt,leftrule=1.5pt]#2
\end{tcolorbox}}

\newcounter{A}[section]
\newcounter{B}[section]
\newcounter{C}[section]
\newcounter{D}[section]
\newcounter{E}[section]
\newcounter{F}[section]

\newcommand{\con}[1]{{\bfseries\refstepcounter{#1}\thesection.\arabic{#1}}}

%\theoremindent0.2cm
%\theoremheaderfont{\CJKfamily{songbold}}
%\theoremstyle{break}

%\newtheorem*{theorem}{\hspace{-0.16cm}\color{2a7ae2}\mybox[2a7ae2]{\color{2a7ae2}定理\addtocounter{A}{1} \thesection.\arabic{A}}}
%\newtheorem*{definition}{\hspace{-0.16cm}\color{f8766d}\mybox[f8766d]{\color{f8766d}定义\addtocounter{B}{1} \thesection.\arabic{B}}}
%\newtheorem*{feature}{\hspace{-0.16cm}\color{ffa725}\mybox[ffa725]{\color{ffa725}性质\addtocounter{C}{1} \thesection.\arabic{C}}}
%\newtheorem*{inference}{\hspace{-0.16cm}\color{1a9850}\mybox[1a9850]{\color{1a9850}推论\addtocounter{D}{1} \thesection.\arabic{D}}}
%\newtheorem*{method}{\hspace{-0.16cm}\color{6a3d9a}\mybox[6a3d9a]{\color{6a3d9a}方法\addtocounter{E}{1} \thesection.\arabic{E}}}
%\newtheorem*{example}{\hspace{-0.16cm}\color{53a9ab}\mybox[53a9ab]{\color{53a9ab}例题\addtocounter{F}{1} \thesection.\arabic{F}}}

\newcommand{\theorem}[1][]{\vspace{1.2em}\par\noindent\hspace{-8pt}\mybox[2a7ae2]{\color{2a7ae2}{\CJKfamily{heiti}定理\ \con{A}\hspace{1em}{\CJKfamily{songbold}#1} } }\vspace{0.5em}\par }
\newcommand{\definition}[1][]{\vspace{1.2em}\par\noindent\hspace{-8pt}\mybox[53a9ab]{\color{53a9ab}{\CJKfamily{heiti}定义\ \con{B}\hspace{1em}{\CJKfamily{songbold}#1} } }\vspace{0.5em}\par }
\newcommand{\feature}[1][]{\vspace{1.2em}\par\noindent\hspace{-8pt}\mybox[ffa725]{\color{ffa725}{\CJKfamily{heiti}性质\ \con{C}\hspace{1em}{\CJKfamily{songbold}#1} } }\vspace{0.5em}\par }
\newcommand{\inference}[1][]{\vspace{1.2em}\par\noindent\hspace{-8pt}\mybox[1a9850]{\color{1a9850}{\CJKfamily{heiti}推论\ \con{D}\hspace{1em}{\CJKfamily{songbold}#1} } }\vspace{0.5em}\par }
\newcommand{\method}[1][]{\vspace{1.2em}\par\noindent\hspace{-8pt}\mybox[6a3d9a]{\color{6a3d9a}{\CJKfamily{heiti}方法\ \con{E}\hspace{1em}{\CJKfamily{songbold}#1} } }\vspace{0.5em}\par }
\newcommand{\example}[1][]{\vspace{1.2em}\par\noindent\hspace{-8pt}\mybox[f8766d]{\color{f8766d}{\CJKfamily{heiti}例题\ \con{F}\hspace{1em}{\CJKfamily{songbold}#1} } }\vspace{0.5em}\par }

%符号设置-------------------------------
{\color{titlepurple}}

%标题配置—————————————
\title{
	% \begin{center}
	 	\includegraphics[width=16cm]{picture/title/covertitle.eps}
	% 	%\includegraphics[width=8cm]{cover2.png}
	% 	%\includegraphics[width=8cm]{cover3.png}
	% 	%\includegraphics[width=12cm]{cover4.png}
	% \end{center}
	\vspace{5cm}}
\author{\large\color{titlepurple}易鹏\ 关舒文\\}
\date{\color{titlepurple}\small{Latest Update\ :\ \today}}

%章节或标题的样式-------------------
\titleformat{\chapter}{\CJKfamily{title}\huge\color{titlepurple}}{\includegraphics[height=2cm]{picture/title/sigmaformal.eps} 第\ \thechapter\ 章\ }{0pt}{}
\titleformat{\section}{\CJKfamily{hei}\large\color{titlepurpleb}}{\bfseries{\thesection}\quad  }{0pt}{}
\titleformat{\subsection}{\CJKfamily{hei}\large\color{titlepurplec}}{\bfseries{\thesubsection}\quad }{0pt}{}
\titlespacing*{\subsection}{0pt}{12pt}{0pt}
\titleformat{\subsubsection}{\CJKfamily{hei}\normalsize\color{titlepurplec}}{\bfseries{\thesubsubsection}\quad  }{0pt}{}
\renewcommand{\nomname}{符号说明}

%自定义优化命令-----------------------
\renewcommand{\a}{\ensuremath A}%
\renewcommand{\b}{\ensuremath B}%
\renewcommand{\c}{\ensuremath C}%
\renewcommand{\d}{\, \, \text{d}}
\newcommand{\e}{\text{e}}
\renewcommand{\o}{\ensuremath \varnothing}%
\renewcommand{\textbf}[1]{{\CJKfamily{heiti}#1}}%
\newcommand{\md}[1]{{\,\color{purple}#1\,}}%
\newcommand{\cf}[1]{\textit{cf.Fig~\ref{#1}}}%
\newcommand{\margin}[1]{{\marginpar{\footnotesize\CJKfamily{kaishu}\hspace*{2em}#1}}}
\newcommand{\f}{\ensuremath \bm{F}}
\newcommand{\smallboxa}[2][]{\tcbox[colframe =Chocolate , colback =Coral,boxrule=0.5mm,size=small,on line]{\color{white}{\textbf{#1}}}  \index{#2@#1}\quad }
\newcommand{\smallboxb}[1][]{\tcbox[colframe =Chocolate, colback =Coral,boxrule=0.5mm,size=small,on line]{\color{white}{\textbf{#1}}} \quad }
\newcommand{\empha}[1][]{\colorbox{文字底色}{#1}}
%\renewcommand{\chapter}[1]{{\chapter{#1}\thispagestyle{empty}} }%

%使用了自定义页眉页脚---------------
\pagestyle{fancy}
\renewcommand{\chaptermark}[1]{\markboth{\;第\ \thechapter\ 章\quad#1\;}{}}
\renewcommand{\sectionmark}[1]{\markright{\;\thesection\ #1\;}}
\fancyhf{}
%\fancyfoot[C]{\bfseries\thepage}
\fancyhead[LO]{\small\CJKfamily{heilight}\rightmark}
\fancyhead[RE]{\small\CJKfamily{heilight}\leftmark}
\fancyhead[RO,LE]{\;\thepage\;}
\fancyfoot[RO,LE]{\small\CJKfamily{heilight}{工程力学}}
\fancyfoot[RE,LO]{\small\CJKfamily{heilight}Engineering mechanics}
\renewcommand{\headrulewidth}{0.4pt} % 注意不用\setlength
%\renewcommand{\footrulewidth}{0pt}
\fancyheadoffset[LE,RO]{4cm}
\fancyfootoffset[LE,RO]{4cm}

%正文部分—————————————
\begin{document}


%目录与公式编号生成——————————
\numberwithin{equation}{section}
\allowdisplaybreaks%强制自动换行
\newgeometry{left=2cm,right=2cm,marginparwidth=0cm,marginparsep=0cm}%封面设置
\pagenumbering{roman}
\maketitle
\thispagestyle{empty}

%页面重新配置----------------------------
\restoregeometry
{\printnomenclature
    \setcounter{page}{0}\pagenumbering{roman}
    \addcontentsline{toc}{chapter}{符号说明}}
\newpage
\pagenumbering{Roman}
\setcounter{page}{0}
\tableofcontents

%色块--------------------------------------
\nomenclature{\colorbox{f8766d}{\qquad} }{Color \texttt{\#EA7500}}
\nomenclature{\colorbox{1a9850}{\qquad}}{Color \texttt{\#0097e6}}
\nomenclature{\colorbox{ffa725}{\qquad}}{Color \texttt{\#1289A7}}
\nomenclature{\colorbox{2a7ae2}{\qquad} }{Color \texttt{\#7158e2}}
\nomenclature{\colorbox{6a3d9a}{\qquad}}{Color \texttt{\#ED4C67}}
\nomenclature{\colorbox{53a9ab}{\qquad}}{Color \texttt{\#007500}}
\nomenclature{\colorbox{titlepurple}{\qquad }}{Color \texttt{\#5758BB}}
\nomenclature{\colorbox{titlepurpleb}{\qquad}}{Color \texttt{\#833471}}
\nomenclature{\colorbox{titlepurplec}{\qquad}}{Color \texttt{\#006266}}
\nomenclature{\colorbox{md}{\qquad}}{Color \texttt{\#EA2027}}
\nomenclature{\colorbox{purple}{\qquad}}{Color \texttt{\#bf0040}}
\nomenclature{\colorbox{文字底色}{\qquad}}{Color \texttt{\#F8FF00}}

%正文开始—————————————---------------------------------------------------------------------------------------可以使用\boldmath输入粗斜体与\unboldmath合用
\chapter{静力学基础}
\pagenumbering{arabic}
\setcounter{page}{1}
% 静力学第一章
\section{静力学基础}
\subsection{力及其性质}
\begin{definition}[力的定义]
    力是物体间的\textbf{相互机械作用},具有两种\md{作用效应}:
    \begin{itemize}
        \item 外效应(运动效应): 改变物体的运动状态;
        \item 内效应(变形效应): 使物体的几何形状或尺寸发生改变.
    \end{itemize}
\end{definition}
\begin{definition}[力的三要素]
    我们常用一个矢量$\bm F$来表示一个力.其中力的三要素:大小,方向,作用点.我们分别用适量的比例长度来表示力的大小,矢量方向表示力的方向,矢量的始端作为作用点.
\end{definition}
\begin{definition}[力系的概念]
    力系: 作用在物体上的一群力;
    平衡力系: 物体在力系作用下处于平衡;
    刚体: 在力的作用下,形状大小和尺寸变化都可以忽略的物体;
    平衡: 物体处于静止或匀速运动状态.
\end{definition}

\begin{theorem}公理
    作用在物体上同一点的两个力,可以合成为一个力,合力作用在同一点,其大小由以这两个力为边的平行四边形的对角线来确定.即合力矢等于这两个力的矢量和,即\[\bm{F}_R=\bm{F}_1+\bm{F}_2\],如图所示


    复杂力系简化的理论基础
\end{theorem}
\begin{inference}[力的多边形法则]
    作用于物体上同一点的多个力(汇交力系),可以合成为一个合力,合力作用在该汇交点上,其大小和方向等于各力的矢量和:
    \[\f_r=\f_1+\f_2+\cdots+\f_n=\sum \f_i\]

\end{inference}

\begin{theorem}[二力平衡条件]
    作用在刚体上的两个力,使物体平衡有以下充要条件:\[\f_1=-\f_2.\]这表明两个力
    \begin{enumerate}
        \item 大小相等
        \item 方向相反
        \item 作用在同一直线上
    \end{enumerate}
    注意:对于刚体来说,上面的条件为充要条件,但是对于变形体和多体,只是必要条件.
\end{theorem}
\begin{definition}
    只在两个力的作用下平衡的构件称为\textbf{二力构件}
\end{definition}

\begin{theorem}[加减平衡力系原理]
    在已知力系上加上或减去任意一个平衡力系,并不改变原力系对刚体的作用效应
\end{theorem}
\begin{inference}[刚体力的可传性]



\begin{theorem}[作用力和反作用力定律]
    作用力与反作用力总是同时存在,大小相等,方向相反,沿着同一条直线,分别作用在两个相互作用的的物体上
\end{theorem}

\chapter{材料力学的基本概念}
\thispagestyle{empty}
\section{基本概念与主要研究任务}
\subsection{基本概念}
\vspace{1em}
\smallboxa[构件]{GJ}组成结构物和机械的最基本的部件.\\[0.5em]
\smallboxa[失效]{SX}工程构件在外力作用下丧失正常功能的现象称为失效或破坏.
\begin{itemize}
	\item 强度失效\index{SX@失效!QDSX@强度失效}\quad 构件在外力作用下发生不可恢复的塑性变形或发生断裂.
	\item 稳定性失效\index{SX@失效!WDXSX@稳定性失效}\quad 构件在外力作用下其平衡形式发生突然转变.(压缩载荷大于某值)
	\item 刚度失效\index{SX@失效!GDSX@刚度失效}\quad 构件在外力作用下产生过量的弹
	性变形。
	
\end{itemize}

\subsection{主要研究任务}
\vspace{1em}
\noindent 1. 四方面要求
\begin{itemize}
	\item 强度\index{QD@强度}要求\quad 杆件在外力作用下绝对不能发生破坏,以避免经济损失和事故发生.
	\item 刚度\index{GD@刚度}要求\quad 杆件在外力作用下不发生过分变形,以保证结构或机器正常工作.
	\item 稳定性\index{WDX@稳定性}要求\quad 杆件在外力作用下不发生过分变形,以保证结构或机器正常工作.
\end{itemize}


\chapter{拉伸与压缩}
\thispagestyle{empty}
\section{轴向拉伸与压缩的概念}
\subsection{变形特点和受力特点}
作用于杆件两端的外力合力的作用线与杆件轴线重合,杆件变形是沿轴线方向的伸长或缩短.

\subsection{截面法求内力}
\begin{method}[截面法\index{JMF@截面法}求内力]
	\begin{itemize}
		\item 假想沿$m-m$截面截开.
		\item 留下左半段或右半段.
		\item 将弃去部分对留下部分的作用用内力代替
		\item 对留下部分写力平衡方程,求出内力的值.
	\end{itemize}
\end{method}

\begin{definition}[轴\index{ZL@轴力}力]
	\margin{轴力的正负性:\\ \empha[拉$\to$正,压$\to$负]}
	由于外力作用线与杆件轴线重合,内力的作用线也与杆件的轴线重合.\\
\end{definition}

\begin{definition}[轴力图\index{ZLT@轴力图}]
	轴力沿杆件轴线的变化.
\end{definition}
\section{轴向拉伸或压缩时横截面上的内力和应力} 
杆件的强度不仅与轴力有关,还与横截面面积有关.必须用应力来比较和判断杆件的强度.在拉(压)杆的横截面上,与轴力$F_ {\n}$对应的应力是正应力$\sigma$.根据连续性假设,横截面上到处都存在着内力.于是得静力关系:
\begin{equation}
	123
\end{equation}




















\chapter{扭转}
\thispagestyle{empty}
\section{扭转的概念}
\vspace*{-1.5em}
\begin{definition}[扭转变形]
	扭转变形是杆件受到大小相等,方向相反且作用平面垂直于杆件轴线的力偶作用,使杆件的横截面绕轴线产生转动。
	\begin{itemize}
		\item 受力特点
		\par 杆件的两端作用两个\md{大小相等}、\md{方向相反}、且\md{作用面垂直于杆件轴线}的力偶.
		\item 变形特点
		\par 杆件的任意两个横截面都发生绕轴线的相对转动.
	\end{itemize}
	\end{definition}

\section{外力偶矩的计算}
\vspace*{-1.5em}
\begin{theorem}[外力偶矩]
	工程中有许多传递功率的轴,需要根据它的转速$n$和传递的功率$N_p$计算出外力偶矩。力偶在单位时间内所作之功就是功率,它等于:
\begin{equation}
	N_p = M_n \omega
\end{equation}
$N_p$常用$\text{kw}$(千瓦)表示,而$w$常用$\text{rpm}$(转/分)表示.即
\margin{\\$M_\text{e}$ \quad 外力偶矩($\text{N}/\text{m}$)\\$P$ \quad 功率($\text{kw}$)\\$n$ \quad 转速($\text{r/min}$)}
\begin{equation}
	M_\text{e}=9549\,\frac{P}{n}
\end{equation}
\end{theorem}

\section{扭矩和扭矩图}















%打印索引—————————————
\newpage
\newgeometry{left=2cm,right=2cm,bottom=2cm,top=2cm,marginparwidth=0cm,marginparsep=0cm}%索引设置
\pagestyle{headings}
\addcontentsline{toc}{chapter}{附录}
\addcontentsline{toc}{section}{索引}
\appendix
\CJKfamily{kaishu}
\color{titlepurple}
\printindex
%———————————————
\end{document}