% 静力学第一章
\section{静力学基础}
\subsection{力及其性质}
\begin{definition}[力的定义]
    力是物体间的\textbf{相互机械作用},具有两种\md{作用效应}:
    \begin{itemize}
        \item 外效应(运动效应): 改变物体的运动状态;
        \item 内效应(变形效应): 使物体的几何形状或尺寸发生改变.
    \end{itemize}
\end{definition}
\begin{definition}[力的三要素]
    我们常用一个矢量$\bm F$来表示一个力.其中力的三要素:大小,方向,作用点.我们分别用适量的比例长度来表示力的大小,矢量方向表示力的方向,矢量的始端作为作用点.
\end{definition}
\begin{definition}[力系的概念]
    力系: 作用在物体上的一群力;
    平衡力系: 物体在力系作用下处于平衡;
    刚体: 在力的作用下,形状大小和尺寸变化都可以忽略的物体;
    平衡: 物体处于静止或匀速运动状态.
\end{definition}

\begin{theorem}公理
    作用在物体上同一点的两个力,可以合成为一个力,合力作用在同一点,其大小由以这两个力为边的平行四边形的对角线来确定.即合力矢等于这两个力的矢量和,即\[\bm{F}_R=\bm{F}_1+\bm{F}_2\],如图所示


    复杂力系简化的理论基础
\end{theorem}
\begin{inference}[力的多边形法则]
    作用于物体上同一点的多个力(汇交力系),可以合成为一个合力,合力作用在该汇交点上,其大小和方向等于各力的矢量和:
    \[\f_r=\f_1+\f_2+\cdots+\f_n=\sum \f_i\]

\end{inference}

\begin{theorem}[二力平衡条件]
    作用在刚体上的两个力,使物体平衡有以下充要条件:\[\f_1=-\f_2.\]这表明两个力
    \begin{enumerate}
        \item 大小相等
        \item 方向相反
        \item 作用在同一直线上
    \end{enumerate}
    注意:对于刚体来说,上面的条件为充要条件,但是对于变形体和多体,只是必要条件.
\end{theorem}
\begin{definition}
    只在两个力的作用下平衡的构件称为\textbf{二力构件}
\end{definition}

\begin{theorem}[加减平衡力系原理]
    在已知力系上加上或减去任意一个平衡力系,并不改变原力系对刚体的作用效应
\end{theorem}
\begin{inference}[刚体力的可传性]



\begin{theorem}[作用力和反作用力定律]
    作用力与反作用力总是同时存在,大小相等,方向相反,沿着同一条直线,分别作用在两个相互作用的的物体上
\end{theorem}