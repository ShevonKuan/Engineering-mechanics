\section{轴向拉伸与压缩的概念}
\subsection{变形特点和受力特点}
作用于杆件两端的外力合力的作用线与杆件轴线重合,杆件变形是沿轴线方向的伸长或缩短.

\subsection{截面法求内力}
\begin{method}[截面法\index{JMF@截面法}求内力]
	\begin{itemize}
		\item 假想沿$m-m$截面截开.
		\item 留下左半段或右半段.
		\item 将弃去部分对留下部分的作用用内力代替
		\item 对留下部分写力平衡方程,求出内力的值.
	\end{itemize}
\end{method}

\begin{definition}[轴\index{ZL@轴力}力]
	\margin{轴力的正负性:\\ \empha[拉$\to$正,压$\to$负]}
	由于外力作用线与杆件轴线重合,内力的作用线也与杆件的轴线重合.\\
\end{definition}

\begin{definition}[轴力图\index{ZLT@轴力图}]
	轴力沿杆件轴线的变化.
\end{definition}
\section{轴向拉伸或压缩时横截面上的内力和应力} 

















