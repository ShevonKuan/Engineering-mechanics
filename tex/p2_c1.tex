\section{基本概念与主要研究任务}
\subsection{基本概念}
\vspace{1em}
\smallboxa[构件]{GJ}组成结构物和机械的最基本的部件.\\[0.5em]
\smallboxa[失效]{SX}工程构件在外力作用下丧失正常功能的现象称为失效或破坏.
\begin{itemize}
	\item 强度失效\index{SX@失效!QDSX@强度失效}\quad 构件在外力作用下发生不可恢复的塑性变形或发生断裂.
	\item 稳定性失效\index{SX@失效!WDXSX@稳定性失效}\quad 构件在外力作用下其平衡形式发生突然转变.(压缩载荷大于某值)
	\item 刚度失效\index{SX@失效!GDSX@刚度失效}\quad 构件在外力作用下产生过量的弹
	性变形。
	
\end{itemize}

\subsection{主要研究任务}
\vspace{1em}
\noindent 1. 四方面要求
\begin{itemize}
	\item 强度\index{QD@强度}要求\quad 杆件在外力作用下绝对不能发生破坏,以避免经济损失和事故发生.\\
	\empha[在外力作用下构件抵抗破坏的能力.]
	\item 刚度\index{GD@刚度}要求\quad 杆件在外力作用下不发生过分变形,以保证结构或机器正常工作.\\
	\empha[在外力作用下构件抵抗变形的能力.]
	\item 稳定性\index{WDX@稳定性}要求\quad 杆件在外力作用下不发生过分变形,以保证结构或机器正常工作.\\
	\empha[在外力作用下保持其原有的平衡状态的能力.]
	\item 经济要求 \quad 在符合安全性要求(前面三条)的情况下尽可能的节省材料.
\end{itemize}

\noindent 2. 主要思路
\margin{注意:与静力学不同的是,由于力的位置对材料的变形有影响,导致内力的变化.因此,在材料力学中,力不可以随意移动.}
\begin{center}
	\begin{tikzpicture}[node distance=1.2cm]
		%定义流程图具体形状
		\node (A) [minimum height=0cm,draw, node distance=1cm,inner sep=6pt] { 变形 };
		\node (B) [minimum height=0cm,draw, below of=A,node distance=0cm,inner sep=6pt,xshift =3.5cm] {应变分布};
		\node (C) [minimum height=0cm,draw, below of=B,node distance=0cm,inner sep=6pt,xshift =4cm] {应力分布};
		\node (D) [minimum height=0cm,draw, below of=C,node distance=0cm,inner sep=6pt,xshift =4cm] {应力公式};
		
		%连接具体形状
		\draw[arrows={-Stealth[scale=0.8]}](A) -- (B) node[midway,left=0.1cm,above=0cm]{平面假设} ;
		\draw[arrows={-Stealth[scale=0.8]}](B) -- (C) node[midway,left=0.1cm,above=0cm]{物理关系} ;
		\draw[arrows={-Stealth[scale=0.8]}](C) -- (D)node[midway,left=0.1cm,above=0cm]{静力方程} ;
	\end{tikzpicture}
\end{center}

\section{基本假设}
\begin{enumerate}[]
	\item\smallboxa[连续性假设]{LXXJS}\\[0.5em]
	\hspace*{2em}认为整个物体体积内毫无空隙地充满物质.
	\item\smallboxa[均匀性假设]{JYXJS}\\[0.5em]
	\hspace*{2em}认为物体内的任何部分,其力学性能相同.
	\item\smallboxa[各向同性假设]{GXTXJS}\\[0.5em]
	\hspace*{2em} 认为在物体内各个不同方向的力学性能相同
	\item\smallboxa[小变形假设]{XBXJS}\\[0.5em]
	\hspace*{2em}变形与本身的尺寸相比很小,可忽略不计.
	\item\smallboxa[线弹性假设]{XTXXJS}\\[0.5em]
	\hspace*{2em}外力与变形成正比.
\end{enumerate}

\section{内力的求法}

\section{应变}
\vspace{-1.5em}
\begin{definition}[应力]
单位面积上的内力\index{YL@应力},即某截面处内力的密集程度.\\
\md{平均应力}\margin{\\ $F$ \quad 总内力大小(N)\\$A$ \quad 截面的面积($\text{m}^2$)\\$p\,$ \quad 截面的应力(Pa) }
\begin{equation}
	p_m = \frac{\Delta F}{\Delta A}
\end{equation}
\md{单点应力}
\begin{equation}
	p = \lim\limits_{\Delta A \to 0}\frac{\Delta F}{\Delta A}=\frac{\d F}{\d A}
\end{equation}
\end{definition}

\begin{definition}[变形]
	构件受力以后\index{BX@变形},形状和尺寸的变化.
	\margin{\\ $l_1$ \quad 变形后长度(m)\\$l_0$ \quad 变形前长度(m)\\$\varepsilon\,\,$ \quad 线应变(无量纲) \\ $\alpha\,$ \quad 变形后角度(rad)\\$\gamma\,\,$ \quad 切应变(无量纲) }
	\\
	\md{线变形}(\md{线应变}\index{XYB@线应变})
	\begin{equation}
		\varepsilon = \frac{l_1-l_0}{l_0}=\frac{\Delta l}{l_0}
	\end{equation}
	\md{角变形}(\md{角应变},\md{切应变}\index{JYB@角应变}\index{QYB@切应变})
	\begin{equation}
		\gamma = \frac{\pi}{2} - \alpha
	\end{equation}
\end{definition}

\section{杆件变形的基本形式}
\begin{enumerate}[]
	\item\smallboxa[二力杆]{ELG}\\[0.5em]
	\hspace*{2em}特点:大小相等,方向相反,作用线与杆重合的力.
	\item\smallboxa[剪切]{JQ}\\[0.5em]
	\hspace*{2em}特点:大小相等,方向相反,相互平行的力.
	\item\smallboxa[扭转]{NJ}\\[0.5em]
	\hspace*{2em}特点:大小相等,方向相反,作用面都垂直于杆件轴线的两个力偶
	\item\smallboxa[弯曲]{WQ}\\[0.5em]
	\hspace*{2em}特点:垂直于杆件轴线的纵向平面内的一对大小相等,方向相反的力偶.
\end{enumerate}


