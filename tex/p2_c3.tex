\section{扭转的概念}
\vspace*{-1.5em}
\begin{definition}[扭转变形]
	扭转变形是杆件受到大小相等,方向相反且作用平面垂直于杆件轴线的力偶作用,使杆件的横截面绕轴线产生转动。
	\begin{itemize}
		\item 受力特点
		\par 杆件的两端作用两个\md{大小相等}、\md{方向相反}、且\md{作用面垂直于杆件轴线}的力偶.
		\item 变形特点
		\par 杆件的任意两个横截面都发生绕轴线的相对转动.
	\end{itemize}
	\end{definition}

\section{外力偶矩的计算}
\vspace*{-1.5em}
\begin{theorem}[外力偶矩]
	工程中有许多传递功率的轴,需要根据它的转速$n$和传递的功率$N_p$计算出外力偶矩。力偶在单位时间内所作之功就是功率,它等于:
\begin{equation}
	N_p = M_n \omega
\end{equation}
$N_p$常用$\text{kw}$(千瓦)表示,而$w$常用$\text{rpm}$(转/分)表示.即
\margin{\\$M_\text{e}$ \quad 外力偶矩($\text{N}/\text{m}$)\\$P$ \quad 功率($\text{kw}$)\\$n$ \quad 转速($\text{r/min}$)}
\begin{equation}
	M_\text{e}=9549\,\frac{P}{n}
\end{equation}
\end{theorem}

\section{扭矩和扭矩图}













